% LaTeX Curriculum Vitae Template
%
% Copyright (C) 2004-2009 Jason Blevins <jrblevin@sdf.lonestar.org>
% http://jblevins.org/projects/cv-template/
%
% You may use use this document as a template to create your own CV
% and you may redistribute the source code freely. No attribution is
% required in any resulting documents. I do ask that you please leave
% this notice and the above URL in the source code if you choose to
% redistribute this file.

\documentclass[11pt,letterpaper]{article}

\usepackage{hyperref}
\usepackage{geometry}
\usepackage[T1]{fontenc}

% Comment the following line to use the default Computer Modern font
% instead of the Palatino font provided by the mathpazo package.
% Remove the 'osf' bit if you don't like the old style figures.
\usepackage[sc,osf]{mathpazo}
\linespread{1.05}
\usepackage{microtype}
% In practice, I use the following font packages instead of mathpazo.
% beramono provides a nice fixed-width font. xagaramon uses the
% (commercial) Adobe Garamond font.
%\usepackage[scaled=0.75]{beramono}
%\usepackage[osf]{xagaramon}

% Set your name here
\def\name{Paul D. Johnson}

% Replace this with a link to your CV if you like, or set it empty
% (as in \def\footerlink{}) to remove the link in the footer:
\def\footerlink{}

% The following metadata will show up in the PDF properties
\hypersetup{
  colorlinks = true,
  urlcolor = black,
  pdfauthor = {\name},
  pdfkeywords = {},
  pdftitle = {\name: Curriculum Vitae},
  pdfsubject = {Curriculum Vitae},
  pdfpagemode = UseNone
}

\geometry{
  body={6.5in, 8.5in},
  left=1.0in,
  top=1.25in
}

% Customize page headers
\pagestyle{myheadings}
\markright{\name}
\thispagestyle{empty}

% Custom section fonts
\usepackage{sectsty}
\sectionfont{\rmfamily\mdseries\Large}
\subsectionfont{\rmfamily\mdseries\itshape\large}

% Other possible font commands include:
 %\ttfamily for teletype,
% \sffamily for sans serif,
% \bfseries for bold,
% \scshape for small caps,
% \normalsize, \large, \Large, \LARGE sizes.

% Don't indent paragraphs.
\setlength\parindent{0em}

% Make lists without bullets
\renewenvironment{itemize}{
  \begin{list}{}{
    \setlength{\leftmargin}{1.5em}
  }
}{
  \end{list}
}

\begin{document}

% Place name at left
{\huge \name}

% Alternatively, print name centered and bold:
%\centerline{\huge \bf \name}

\vspace{0.25in}

\begin{minipage}[t]{0.5\textwidth}
School of Mathematics and Statistics \\
University of Sheffield \\
Western Bank \\
Sheffield S10 2TN, UK
\end{minipage}
\begin{minipage}[t]{0.5\textwidth}
  Date of Birth: December 20, 1980 \\
  Citizenship: United States \\
  Email: paul.johnson@sheffield.ac.uk
  \end{minipage}

\section*{Research Interests}
\begin{itemize}
\item Algebraic geometry and combinatorics, in particular Gromov-Witten theory and Donaldson-Thomas theory of orbifolds and the combinatorics of partitions.
\end{itemize}

\section*{Education}
\begin{tabular}{rl}
   2006--2009 & University of Michigan \\
  & Ph.D. in Mathematics \\
  & Advisor: Yongbin Ruan \\
  \rule{0pt}{3ex}2003--2006 &  University of Wisconsin-Madison\\
  & M.A. in Mathematics \\
  \rule{0pt}{3ex}1999--2003 &  University of Chicago \\
  & B.A. in Mathematics \\


\end{tabular}


\section*{Professional History}
\begin{tabular}{rl}
2014-- & Lecturer, University of Sheffield \\
\rule{0pt}{3ex} 2013--2014 & Assistant Professor, Colorado State University \\
\rule{0pt}{3ex} 2011--2013 & NSF Postdoctoral Fellow, Columbia University \\
\rule{0pt}{3ex}2009 -- 2011 & Postdoctoral Research Fellow, Imperial College London \\
\rule{0pt}{3ex} Summer 2009 & NSF Postdoctoral Fellow, Princeton University
\end{tabular}



\section*{Papers }
\begin{enumerate}
\item Double Hurwitz numbers via the infinite wedge. \\
\emph{Trans. Amer. Math. Soc.} 367 (2015), no. 9, 6415--6440. \href{http://arxiv.org/abs/1008.3266}{arXiv:1008.3266}.

\item Equivariant GW Theory of Stacky Curves \\  \emph{Comm. Math. Phys.} 327 (2014), no. 2, 333--386. \href{http://arxiv.org/abs/0903.1068}{arXiv:0903.1068}
\item The quantum Lefschetz hyperplane principle can fail for positive orbifold hypersurfaces. \\
 Joint with T. Coates, A. Gholampour, H. Iritani, Y. Jiang and C. Manolache. \\
   \emph{Mathematical Research Letters} 19 (2012), no 5. 997--1005.
\href{http://arxiv.org/abs/1202.2754}{arXiv:1202.2754}.
\item Hurwitz numbers, ribbon graphs, and tropicalization. \\ Tropical
  geometry and integrable systems, 55--72,\emph{Contemp. Math.}, 580,
  Amer. Math. Soc., Providence, RI, 2012.
\href{http://arxiv.org/abs/1303.1543}{arXiv:1303.1543}.
\item Chamber Structure for double Hurwitz numbers.
\\ Joint with R. Cavalieri and H. Markwig. \\ \emph{Adv. Math.} 228 (2011), no. 4,
  1894--1937.
\href{http://arxiv.org/abs/1003.1805}{arXiv:1003.1805}.
 \item Abelian Hurwitz-Hodge integrals. \\ 
Joint with R. Pandharipande and H.-H. Tseng.  \\  
\emph{Michigan Math. J.} 60 (2011), no. 1, 171--198.
\href{http://arxiv.org/abs/0803.0499}{arXiv:0803.0499}.
\item Tropical Hurwitz Numbers. \\
 Joint with R. Cavalieri and H. Markwig. \\
\emph{J. Algebraic Combin.} 32 (2010), no. 2, 241--265. 
\href{http://arxiv.org/abs/0804.0579}{arXiv:0804.0579}.


\end{enumerate}



\section*{Preprints}
\begin{enumerate}

\item Lattice points and simultaneous core partitions. \\
\href{http://arxiv.org/abs/1502.07934}{arXiv:1502.07934}.

\item A graphical interface for the Gromov-Witten theory of curves. \\
 Joint with R. Cavalieri, H. Markwig, and D. Ranganathan. \\
\href{http://arxiv.org/abs/1604.07250}{arXiv:1604.07250}.

\end{enumerate}



\section*{Teaching Experience}
\begin{tabular}{rl}
University of Sheffield & Lectured Graph Theory and Algebraic Geometry \\
& Ran discussion sections for flipped engineering math \\
\rule{0pt}{3ex}Colorado State University & Combinatorics, Graduate Algebraic Topology sequence \\
\rule{0pt}{3ex}Imperial College London & Taught representation theory of finite groups for advanced math majors.  \\
& Ran math discussion sections for 1st year electrical engineering majors. \\
\rule{0pt}{3ex}University of Michigan & Graduate Student Instructor. Taught Calculus I and II to small classes. \\
\rule{0pt}{3ex}University of Wisconsin &  Teaching Assistant. Led discussion sections for Calculus I, Calculus III \\
&  Business Calculus, and Calculus with Precalculus. \\
%\rule{0pt}{3ex}University of Chicago &  Counselor for YSP, a day math camp for high schoolers. \\
%& Assistant for the SESAME program, which teaches courses for \\ & certification in math and science for Chicago Public Schools teachers.\\
\end{tabular}


\section*{Honors, Awards, \& Fellowships}

\begin{itemize}
\item NSF Postdoctoral Fellowship
\item Sumner B. Myers Prize (Best Math PhD thesis at University of Michigan)
\end{itemize}






\end{document}


% Footer
\begin{center}
  \begin{footnotesize}
    Last updated: \today \\
    \href{\footerlink}{\texttt{\footerlink}}
  \end{footnotesize}
\end{center}





AWARDS LEFT OUT
\item RTG Fellowship: University of Michigan, spring and fall 2008.
\item VIGRE Merit Fellowship: University of Wisconsin, summer 2004 and spring 2005.
\item Wirt and Mary Cornwell Prize (for research at University of Michigan)



\section*{Selected Invited Talks}
\begin{itemize}


\item Tropical Day, July 2013, Warwick University
\item Topology and Geometry Seminar, June 2013, Imperial College

\item Algebraic Geometry Seminar, 

\item Tropical Geometry and Integrable Systems Conference, July 2011, University of Glasgow
\item Tropical Geometry Workshop, July 2010, University of G\"ottingen
\item Sumner Meyers Colloquium, March 2010, University of Michigan
\item Integrability and Number Theory Meeting, March 2010,  University of Edinburgh
\item Geometry and Topology Seminar, February 2010, University of Glasgow
\item Integrability Day, November 2009, Loughborough University
\item Algebraic Geometry Seminar, November 2009, University of Grenoble
\item Algebraic Geometry Seminar, October 2009, University of Warwick
\item Algebraic Combinatorics Seminar, September 2009, Colorado State University
\item Geometry and Physics Workshop, August 2009, Sichuan University
\item Geometry, Representation Theory and Moduli Seminar, April 2008, Princeton University.
\item Geometry Seminar, March 2008, Massachusetts Institute of Technology
\end{itemize}
